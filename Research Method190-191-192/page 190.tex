\documentclass{article}
\usepackage[utf8]{inputenc}

\title{E-research methods,strategies,and issuse by Terry Anderson, Heather Kanuka}
\author{Sara Naseri }

\begin{document}

\maketitle

\title{chapter thirteen}

Result as the editor weaves the disparate articles into a theme issue or waits until the right-sized opening appears in the production cycle . Of even greater concern is the inherent conservatism of the process that has been shown to accept refinements of ordinary science and to reject research that is more revolutionary or explores a new paradigm. Readings (1994) rather caustically describes the reviewer's task as follows: "Normally , those who review essays for inclusion in scholarly journals know what they are supposed to do. Their function is to take exciting , innovative , and challenging work by younger scholars and find reasons to reject it."

Until relatively recently , the paper-based peer-reviewed journal was the only venue that researchers had for scholarly publishing. Of course , academics who were early adopters of the Net quickly recognized that the Net could provides a variety of means to improve the production and dissemination of peer-reviewed research results. As the early pioneer editor of psychology content online , Stephen Harnard (1996) notes :

The scholarly communicative potential of electronic networks is revolutionary. There is only one sector in which the Net will have to be traditional , and that is in the validation of scholarly ideas and findings by peer review. Refereeing can be implemented much more rapidly , equitably , and efficiently  on the Net , but it cannot be dispensed with.(p.109)

Although not all researchers agree that the peer-review is the penultimate achievement in dissemination , it is a safe bet that without peer review , electronic publications would not be accepted for hiring , tenure , and/or promotion within academia. Electronic publication without peer review may , however , provide many other benefits to society and individual researchers. These advantages are discussed in the following sections related to personal Web sites and popular press dissemination.


The principle advantages of electronic publishing of peer-reviewed articles include :

\begin{itemize}
\item increased accessibility;
\item improved capacity for search and retrieval;
\item capacity to link to additional content including source data;
\item capacity to publish whenever the review is finished , rather than waiting for a publication date;
\item increased speed of distribution; and
\item the ability to insert hyperlinks and multimedia into the publication;
\end{itemize}

These reasons seem to us to create important advantages over paper-based publications that will eventually lead to the migration of most scholarly journals to an online format. Most obviously , e-journals have the proven capacity to significantly reduce the time for dissemination. For example the Royal society of Chemistry reduced the time of production from 100 days to 40 days using electronic publications (Wilkenson,2000). A more esoteric advantage of Net publications in hypertext format is the capacity to question the epistemological assumptions of scholarly communication.


\end{document}
