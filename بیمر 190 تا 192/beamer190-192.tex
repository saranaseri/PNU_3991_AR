\documentclass{beamer}


\usetheme[useblacktitletext]{diepen}

\usepackage{graphicx}

% Copyright (C) 2018-2020 Pasquale Claudio Africa and the LaTeX community.
% A full list of contributors can be found at
%
%     https://github.com/elauksap/focus-beamertheme
% 
% This file is part of beamerthemefocus.
% 
% beamerthemefocus is free software: you can redistribute it and/or modify
% it under the terms of the GNU General Public License as published by
% the Free Software Foundation, either version 3 of the License, or
% (at your option) any later version.
% 
% beamerthemefocus is distributed in the hope that it will be useful,
% but WITHOUT ANY WARRANTY; without even the implied warranty of
% MERCHANTABILITY or FITNESS FOR A PARTICULAR PURPOSE. See the
% GNU General Public License for more details.
% 
% You should have received a copy of the GNU General Public License
% along with beamerthemefocus. If not, see <http://www.gnu.org/licenses/>.

\mode<presentation>


% DEFINE COLORS. ---------------------------------------------------------------
\definecolor{main}{RGB}{64, 64, 64}
\definecolor{background}{RGB}{239, 239, 239}

\definecolor{alert}{RGB}{180, 0, 0}
\definecolor{example}{RGB}{0, 110, 0}


% SET COLORS. ------------------------------------------------------------------
\setbeamercolor{normal text}{fg=main, bg=background}
\setbeamercolor{alerted text}{fg=alert}
\setbeamercolor{example text}{fg=example}

\setbeamercolor{titlelike}{fg=background, bg=main}
\setbeamercolor{frametitle}{parent={titlelike}}

\setbeamercolor{footline}{fg=background, bg=main}

\setbeamercolor{block title}{bg=main!80!background, fg=background}
\setbeamercolor{block body}{bg=main!10!background, fg=main}

\setbeamercolor{block title alerted}{bg=alert, fg=background}
\setbeamercolor{block body alerted}{bg=alert!10!background, fg=main}

\setbeamercolor{block title example}{bg=example, fg=background}
\setbeamercolor{block body example}{bg=example!10!background, fg=main}

\setbeamercolor{itemize item}{fg=main}
\setbeamercolor{itemize subitem}{fg=main}

\setbeamercolor{enumerate item}{fg=main!70!black}
\setbeamercolor{enumerate subitem}{fg=main!70!black}

\setbeamercolor{description item}{fg=main!70!black}
\setbeamercolor{description subitem}{fg=main!70!black}

\setbeamercolor{caption name}{fg=main}

\setbeamercolor{section in toc}{fg=main}
\setbeamercolor{subsection in toc}{fg=main}
\setbeamercolor{section number projected}{bg=main}
\setbeamercolor{subsection number projected}{bg=main}

\setbeamercolor{bibliography item}{fg=main}
\setbeamercolor{bibliography entry author}{fg=main!70!black}
\setbeamercolor{bibliography entry title}{fg=main}
\setbeamercolor{bibliography entry location}{fg=main}
\setbeamercolor{bibliography entry note}{fg=main}

\mode<all>

\title{E-research methods,strategies,and issuse by Terry Anderson, Heather Kanuka}
\author{Sara Naseeri}


\begin{document}
  {%
    \setbeamertemplate{headline}{}
    \frame{\titlepage}
  }

  \begin{frame}
\title{chapter thirteen}

Result as the editor weaves the disparate articles into a theme issue or waits until the right-sized opening appears in the production cycle . Of even greater concern is the inherent conservatism of the process that has been shown to accept refinements of ordinary science and to reject research that is more revolutionary or explores a new paradigm. Readings (1994) rather caustically describes the reviewer's task as follows: "Normally , those who review essays for inclusion in scholarly journals know what they are supposed to do. Their function is to take exciting , innovative , and challenging work by younger scholars and find reasons to reject it."


  \end{frame}
  
    \begin{frame}
    
Until relatively recently , the paper-based peer-reviewed journal was the only venue that researchers had for scholarly publishing. Of course , academics who were early adopters of the Net quickly recognized that the Net could provides a variety of means to improve the production and dissemination of peer-reviewed research results. As the early pioneer editor of psychology content online , Stephen Harnard (1996) notes :



  \end{frame}
  
    \begin{frame}
  
The scholarly communicative potential of electronic networks is revolutionary. There is only one sector in which the Net will have to be traditional , and that is in the validation of scholarly ideas and findings by peer review. Refereeing can be implemented much more rapidly , equitably , and efficiently  on the Net , but it cannot be dispensed with.(p.109)

  \end{frame}
  
  \begin{frame}
Although not all researchers agree that the peer-review is the penultimate achievement in dissemination , it is a safe bet that without peer review , electronic publications would not be accepted for hiring , tenure , and/or promotion within academia. Electronic publication without peer review may , however , provide many other benefits to society and individual researchers. These advantages are discussed in the following sections related to personal Web sites and popular press dissemination.

  \end{frame}

  \begin{frame}
  
The principle advantages of electronic publishing of peer-reviewed articles include :

\begin{itemize}
\item increased accessibility;
\item improved capacity for search and retrieval;
\item capacity to link to additional content including source data;
\item capacity to publish whenever the review is finished , rather than waiting for a publication date;
\item increased speed of distribution; and
\item the ability to insert hyperlinks and multimedia into the publication;
\end{itemize}

  \end{frame}
  
\begin{frame}

These reasons seem to us to create important advantages over paper-based publications that will eventually lead to the migration of most scholarly journals to an online format. Most obviously , e-journals have the proven capacity to significantly reduce the time for dissemination. For example the Royal society of Chemistry reduced the time of production from 100 days to 40 days using electronic publications (Wilkenson,2000). A more esoteric advantage of Net publications in hypertext format is the capacity to question the epistemological assumptions of scholarly communication.

  \end{frame}
  
\begin{frame}
McKerrow , Wood , ans Smith(1995) point out that "this opening of the text represents an opportunity to reassert that sense of scholarship being a collegial exechange , rather than received knowledge from expert opinion.''Finally , a number of Net-based systems have been developed to improve the peer review process , most notably by speeding up the process , tracking the movement of the article through the review stages , automating key functions and providing increased opportunity for review and comment. For example , Yu and Schmid (1999)  describe a system of intelligent publishing ''agents'' that assists editors , reviewsers , and authors in task such as exchanging information , reminding of deadlines , and selecting appropriate resources .

 \end{frame}
 
 
 
\begin{frame}
An example of a traditional educational journal that disseminates exclusively via the Net is the International Review of Researcb in Open and Distance Learning (see http://www.irrodl.org/). This journal conducts all its reviewing processes online and produces the final reviewed articles on the Web free of charge to all users. The review process is rather traditional , only using the Net for circulation of drafts to blind reviewers and to distribute the final publication.
 \end{frame}
 
 
 
 
 
\begin{frame}
Av much more radical example of the capacity of the Net to expand the review process is provided by the Journal of Interactive Media in Education (JIME) (http://www.jime.open.ac.uk). JIME uses an open review process designed to make the review process more responsive and dynamic. Specifically , the JIME process acknowledges that

 \end{frame}
 
 
\begin{frame}
\begin{itemize}
\item authors have the right of reply ;
\item reviewers are named and accountable for their comments , and their contributions are acknowledged ; and
\item the winder research cimmunity has the chance to shape a submission before publication (via the JIME Web site).
\end{itemize}
 
These principles create a review process that includes opportunity for both expert and public review and alterations by authors before final publication. The process is illustrated in figure 13.1 .
 \end{frame}
 
 
 
\begin{frame}
Unfortunately , the increase in access to peer-reviewed articles at no charge via the Net does not alaways meet the commercial needs of the publish. Thus , there are a variety of means by which access is restricted to those who compensate the publishers in some way. The most common means is to provide access to the online version of a journal only to those subscribers who already subscribe to the paper version of the journal , who pay a personal subscription fee for online access directly to the publisher , or who are affiliated with an  organization that pays asite-license subscription fee. Often users from these organizations are forced to log on through their institution's computer system or through a proxy service that has  authenticated their identity before being allowed access to the full text of these journals.

 \end{frame}
 
 
 
\begin{frame}
 There are also commercial service libraries , such as CARL (http://www.carl.org), that provide access to particular articles or faxed copies on a fee-for-service basis. CARL charges \$10.00 plus the fee from the copyright owner (varies by article) plus the fax costs if the article is not available in electronic format. CARL currently indexes over 18,000 periodicals (both paper and Net-based) amounting to over 440 new citations being added to their database of available articles per day.
 \end{frame}
 
 \begin{frame}
 \begin{figure}[htp]
    \centering
    \includegraphics[width=10cm]{fig13.jpg}
    \caption{ 13.1 Nontraditional Review Process at the Journal of Interactive Media in Education (http://www-jime.open.ac.uk/resources/icons/lifecycle.gif  }
    \label{fig:13.1}
\end{figure}
  \end{frame}
  
   \begin{frame}
   There is a serious debate among academic researchers to the value of publishing in online journals. A significant group of authors contend that e-journals do not offer the permanence nor the prestige of traditional paper journals. There is also disagreement among faculty related to to the thoroughness of review for online publications (Sweeney , 2000). It is true that we feel something quite comforting about being able to hold a paper copy of one of our traditionally published works. However , we believe that this pleasure is due more to familiarity and sentimentality than to real value. Unlike a good novel , most consumers of educational research do not wish to curl up in bed with the latest edition . Since Net-distributed papers can usually be printed if desired , we can anticipate continued evolution from paper and electronic publication.
  \end{frame}
  
   \begin{frame}
   Many periodicals are now publishing both paper and electronic versions , thus providing the benefits of both media. We have published in both electronic and paper journals and find that there is currently not a great deal of difference in interest nor resulting reward or prestige between the two modes. However , we find it is very useful to be able to respond to inquiries by providing a hyperlink URL rather than taking the trouble and expense of photocopying and mailing copies of our work.
  \end{frame}
  
   \begin{frame}
   Determining which format and which particular journal to submit your results to is often a challenging task. Our advice to the prospective author is to review the e-journals available in your discipline and judge for yourself their quality and the compatibility (in terms of style , methodology , typical subject topics , etc.) between your results and that presented in previous articles.

  \end{frame}
 
 

 
 
\end{document}