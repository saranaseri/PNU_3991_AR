\documentclass{article}
\usepackage[utf8]{inputenc}

\title{E-research methods,strategies,and issuse by Terry Anderson, Heather Kanuka}
\author{Sara Naseri }
\date {azar 1399}

\begin{document}

\maketitle

McKerrow , Wood , ans Smith(1995) point out that "this opening of the text represents an opportunity to reassert that sense of scholarship being a collegial exechange , rather than received knowledge from expert opinion.''Finally , a number of Net-based systems have been developed to improve the peer review process , most notably by speeding up the process , tracking the movement of the article through the review stages , automating key functions and providing increased opportunity for review and comment. For example , Yu and Schmid (1999)  describe a system of intelligent publishing ''agents'' that assists editors , reviewsers , and authors in task such as exchanging information , reminding of deadlines , and selecting appropriate resources .

An example of a traditional educational journal that disseminates exclusively via the Net is the International Review of Researcb in Open and Distance Learning (see http://www.irrodl.org/). This journal conducts all its reviewing processes online and produces the final reviewed articles on the Web free of charge to all users. The review process is rather traditional , only using the Net for circulation of drafts to blind reviewers and to distribute the final publication.

Av much more radical example of the capacity of the Net to expand the review process is provided by the Journal of Interactive Media in Education (JIME) (http://www.jime.open.ac.uk). JIME uses an open review process designed to make the review process more responsive and dynamic. Specifically , the JIME process acknowledges that

\begin{itemize}
\item authors have the right of reply ;
\item reviewers are named and accountable for their comments , and their contributions are acknowledged ; and
\item the winder research cimmunity has the chance to shape a submission before publication (via the JIME Web site).
\end{itemize}
 
These principles create a review process that includes opportunity for both expert and public review and alterations by authors before final publication. The process is illustrated in figure 13.1 .

Unfortunately , the increase in access to peer-reviewed articles at no charge via the Net does not alaways meet the commercial needs of the publish. Thus , there are a variety of means by which access is restricted to those who compensate the publishers in some way. The most common means is to provide access to the online version of a journal only to those subscribers who already subscribe to the paper version of the journal , who pay a personal subscription fee for online access directly to the publisher , or who are affiliated with an  organization that pays asite-license subscription fee. Often users from these organizations are forced to log on through their institution's computer system or through a proxy service that has  authenticated their identity before being allowed access to the full text of these journals. There are also commercial service libraries , such as CARL (http://www.carl.org), that provide access to particular articles or faxed copies on a fee-for-service basis. CARL charges \$10.00 plus the fee from the copyright owner (varies by article) plus the fax costs if the article is not available in electronic format. CARL currently indexes over 18,000 periodicals (both paper and Net-based) amounting to over 440 new citations being added to their database of available articles per day.















\end {document}