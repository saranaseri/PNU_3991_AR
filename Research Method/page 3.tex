\documentclass{article}
\usepackage[utf8]{inputenc}
\usepackage{graphicx}

\title{E-research methods,strategies,and issuse by Terry Anderson, Heather Kanuka}
\author{Sara Naseri}
\date{azar 1399}

\begin{document}

\maketitle

\begin{figure}[htp]
    \centering
    \includegraphics[width=12cm]{fig13.jpg}
    \caption{ 13.1 Nontraditional Review Process at the Journal of Interactive Media in Education (http://www-jime.open.ac.uk/resources/icons/lifecycle.gif  }
    \label{fig:13.1}
\end{figure}

There is a serious debate among academic researchers to the value of publishing in online journals. A significant group of authors contend that e-journals do not offer the permanence nor the prestige of traditional paper journals. There is also disagreement among faculty related to to the thoroughness of review for online publications (Sweeney , 2000). It is true that we feel something quite comforting about being able to hold a paper copy of one of our traditionally published works. However , we believe that this pleasure is due more to familiarity and sentimentality than to real value. Unlike a good novel , most consumers of educational research do not wish to curl up in bed with the latest edition . Since Net-distributed papers can usually be printed if desired , we can anticipate continued evolution from paper and electronic publication.

Many periodicals are now publishing both paper and electronic versions , thus providing the benefits of both media. We have published in both electronic and paper journals and find that there is currently not a great deal of difference in interest nor resulting reward or prestige between the two modes. However , we find it is very useful to be able to respond to inquiries by providing a hyperlink URL rather than taking the trouble and expense of photocopying and mailing copies of our work.

Determining which format and which particular journal to submit your results to is often a challenging task. Our advice to the prospective author is to review the e-journals available in your discipline and judge for yourself their quality and the compatibility (in terms of style , methodology , typical subject topics , etc.) between your results and that presented in previous articles.









\end{document}




